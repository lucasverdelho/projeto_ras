\chapter*{9. Requisitos Funcionais}


\section*{Modelação de Requisitos}
{\TextoCor
Para o levantamento de requisitos foi utilizada a \textit{requirement shell} do modelo \textit{Volere} como forma de representação, para os descrever concisamente.
}

\req
    {12} % ID requisito
    {Funcional} % tipo de requisito
    {1, 7, 20} % use cases
    {O produto deve registar todas as estradas que foram repavimentadas} % descricao
    {Para permitir agendar o reparo de estradas não repavimentadas e identificar potenciais perigos} % rational
    {Equipa} % origem
    {O registo das estradas repavimentadas deve seguir a especificação da IP e ser feito até 30 minutos após concluída a repavimentação da estrada} % fit criterion
    {\prioMust} % prioridade
    {Exemplo de especificação de um requisito} % caption


{\TextoCor
Como caracterização da tabela de representação de requisitos, é necessário descrever os campos:
\begin{itemize}
    \item \textbf{Requisito:} número de identificação do requisito.
    \item \textbf{Tipo:} tipo de requisito, considerando o modelo de \textit{Volere}.
    \item \textbf{\textit{Use Cases}:} número dos Use Cases associados.
    \item \textbf{Descrição:} descrição clara e concisa do requisito.
    \item \textbf{\textit{Rationale}:} razão para a existência do requisito.
    \item \textbf{Origem:} quem originou o requisito.
    \item \textbf{\textit{Fit criterion}:} critério para validar cumprimento do requisito.
    \item \textbf{Prioridade:} índice de prioridade para a implementação do requisito:
    \begin{itemize}
        \item \prioMust: requisito obrigatório;
        \item \prioShould: requisitos que deve ser implementados;
        \item \prioCould: requisito que não é necessário, mas é desejado;
        \item \prioWont: requisito que pode ser considerado posteriormente.
    \end{itemize}
    \item \textbf{Data:} data da especificação do requisito.
\end{itemize}
}

\section*{Requisitos Funcionais}

{\TextoCor

\req
    {Req1} % ID requisito
    {Funcional} % tipo de requisito
    {1} % use cases
    {O Docente cria uma prova de avaliação} % descricao
    {Para permitir associar questões que serão respondidas pelos alunos e, posteriormente, classificadas} % rational
    {Cliente} % origem
    {A criação de uma prova implica introduzir toda a informação necessária à sua realização e inequívoca identificação} % fit criterion
    {\prioMust} % prioridade
    {Requisito funcional quanto à criação de uma prova de avaliação} % caption

% FDias: Ver questão da inscrição dos alunos e uniformizar! (comentário na figura dos Diagramas de Use Cases)
\req
    {Req2} % ID requisito
    {Funcional} % tipo de requisito
    {1} % use cases
    {\hl{O Técnico regista Alunos numa prova de avaliação}} % descricao
    {Para permitir definir que Alunos poderão participar na prova, associando-lhes também um método de autenticação. Em alternativa, o Docente também deve poder registar alunos, que não estejam, por alguma razão, ainda inscritos.} % rational
    {Cliente} % origem
    {O processo de registo de alunos deve permitir definir que alunos podem participar na prova, e garantir que cada um deles pode ser autenticado pela plataforma aquando da sua realização} % fit criterion
    {\prioMust} % prioridade
    {Requisito funcional quanto ao registo de alunos numa prova de avaliação} % caption

\req
    {Req3} % ID requisito
    {Funcional} % tipo de requisito
    {1} % use cases
    {O Docente adiciona Questões de escolha múltipla a uma prova de avaliação} % descricao
    {Para permitir avaliar o conhecimento dos Alunos numa dada temática, facultando um conjunto de possíveis respostas, onde apenas uma é a correta} % rational
    {Cliente} % origem
    {As Provas de avaliação podem ser compostas por Questões de escolha múltipla} % fit criterion
    {\prioMust} % prioridade
    {Requisito funcional quanto à possibilidade de criar Questões de escolha múltipla} % caption

\req
    {Req4} % ID requisito
    {Funcional} % tipo de requisito
    {\hl{...}} % use cases
    {O Docente define o momento a partir do qual as provas podem ser consultadas} % descricao
    {Para permitir ao Docente disponibilizar as respostas dadas por cada Aluno, assim como as respetivas classificações} % rational
    {Cliente} % origem
    {As Provas de avaliação apenas podem ser consultadas após o momento definido pelo Docente} % fit criterion
    {\prioShould} % prioridade
    {Requisito funcional quanto à possibilidade de disponibilizar Provas de avaliação para consulta} % caption
}

% \req
%     {5} % ID requisito
%     {Funcional} % tipo de requisito
%     {\hl{...}} % use cases
%     {O produto deve permitir definir o momento a partir do qual as provas podem ser consultadas} % descricao
%     {Para permitir ao docente disponibilizar as respostas dadas por cada aluno, assim como as respetivas classificações} % rational
%     {Cliente} % origem
%     {As provas de avaliação apenas podem ser consultadas após o momento definido pelo docente} % fit criterion
%     {\prioShould} % prioridade
%     {Requisito funcional quanto à possibilidade de disponibilizar provas para consulta} % caption

% FDias: algumas ideias
% - Alunos poderem ter acesso (ou até receber) um PDF com as suas respostas / classificações
% - Prof. poder usar bold, itálico, tamanhos de texto diferentes, etc. nas provas
% - Prof. poder associar regras em regex para validar input dos alunos (se a resposta tiver de ser um número, validar que é efetivamente um número logo no UI qd o aluno está a escrever)
% - Prof. poder editar a prova enquanto ela está a decorrer (algum erro que seja detetado). Não podemos é afetar as respostas já dadas: imutabilidade / versões (?)
% - O UI dos Alunos poder mostrar um cronómetro (configurável pelo aluno [?])
% - Prof. poder dar indicação, junto da questão, o tempo esperado para a resposta àquela questão. Depois apresentar isso ao aluno (tb configurável) (?)
% - Prof. poder adicionar multimedia às questões
% - 
% - ...
% Nota: temos de fazer algumas entrevistas com Profs. e Alunos para percebermos as necessidades.

...
