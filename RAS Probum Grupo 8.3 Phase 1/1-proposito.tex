\chapter*{1. Propósito do Projeto}

\section*{Contexto}



% Crescente digitalização do ensino, nomeadamente pilotos em provas da escolaridade obrigatória portuguesa
% Cursos com um número crecente de alunos inscritos, e UCs com várias provas de avaliação:
% - Complexidade na gestão de espaços para as provas (as salas são insuficientes para o número de alunos, pelo que uma ajuda informática seria uma ajuda muito importante);
% - Ineficiência na correção das provas (recursos e tempo);
% - Despredicio de papel.

{\TextoCor
Nos últimos anos, tem-se assistido a uma crescente integração da tecnologia no ambiente educativo e esse fenómeno não passa despercebido no ensino superior. Com o objetivo de otimizar processos e melhorar a experiência de Alunos e Docentes, muitas organizações têm explorado a digitalização de várias atividades, incluindo a realização de provas de avaliação.

Um exemplo interessante desta tendência é aquele que se observa em algumas provas de aferição do ensino básico português, que tentam explorar o potencial de transformação que as soluções digitais podem proporcionar.
No entanto, aplicar no imediato estes métodos numa instituição de ensino superior levanta novos desafios, que estão essencialmente relacionados com as limitações das estruturas informáticas. O crescente número de Alunos inscritos em muitas unidades curriculares é também um obstáculo à aplicação desses métodos.
Dentro desta realidade, destacam-se os seguintes desafios:

\begin{itemize}
\item \textbf{Gestão de espaços e recursos:} A realização de provas de avaliação para um grande número de Alunos coloca uma pressão significativa sobre os espaços disponíveis. As salas revelam-se muitas vezes insuficientes, resultando em desafios logísticos na calendarização e alocação destes espaços.

\item \textbf{Eficiência na correção:} A correção manual de um grande número de provas é um processo demorado e sujeito a erros humanos. Fazê-la, especialmente em cursos com muitos Alunos, torna-se ineficiente e onerosa em termos de tempo e custo.

\item \textbf{Sustentabilidade ambiental:} O uso de papel na impressão e resposta às provas de avaliação resulta num desperdício muito significativo, que tem um impacto negativo no ambiente. A necessidade de encontrar alternativas mais sustentáveis para a realização das provas é um aspeto crítico.

\end{itemize}
}

\section*{Objetivos do Projeto}
{\TextoCor
Para responder a estes desafios, propomos o desenvolvimento de um produto, o Probum, que tenha algumas características diferenciadoras e até inovadoras, que aborda todos estes problemas, relacionados com a realização de provas de avaliação, de forma integrada. O Probum permitirá:

\begin{itemize}
%----------------------
\item \textbf{Criação de provas de avaliação digitais:} Criar provas de avaliação personalizadas será simples e fácil e permitirá incorporar questões de diferentes tipos, temas e níveis de dificuldade. Deve ser possível criar questões alternativas, com respostas também diferenciadas, se for conveniente cada Aluno ter uma prova diferente das dos restantes colegas.
%----------------------
\item \textbf{Calendarização inteligente das provas:} Utilizando ferramentas matemáticas (e.g., algoritmos de otimização), o Probum poderá gerar automaticamente um calendário para uma dada prova de avaliação, tendo em consideração os espaços disponíveis, o número de Alunos inscritos, e as necessidades específicas de cada unidade curricular ou curso. Se uma dada prova tiver, por exemplo, 96 Alunos inscritos e a sala para a realização dessa prova só tiver capacidade para 20 Alunos, então têm que ser organizadas pelo menos 5 rondas de realização do teste. Este facto pode obrigar a que as provas não sejam iguais para todos os Alunos, como atrás se referiu.  
%----------------------
\item \textbf{Realização das provas:} Um Aluno poderá realizar as provas de avaliação de forma digital, através de uma plataforma dedicada, proporcionando assim uma experiência simples, robusta, e flexível. Esta plataforma será disponibilizada em espaços e equipamentos da própria IES, garantindo todos os requisitos exigidos (e.g., autenticidade, confidencialidade, equidade, duração, plágio). Deve ser analisada e explorada a possibilidade de compaginar a realização de provas em computadores da IES com provas realizadas em computadores dos Alunos, desde que se mantenham todas as garantias de confidencialidade e autenticidade.
%----------------------
\item \textbf{Correção:} Na fase de correção, todas as questões de resposta fechada serão facilmente avaliadas e pontuadas de forma automática. O Probum deverá poder ser estendido com componentes baseados em técnicas de processamento de linguagem natural, para auxiliar a correção de questões de resposta livre. Se tal funcionalidade não estiver disponível, cabe ao Docente pontuar essas questões abertas.
%----------------------
\item \textbf{Sustentabilidade:} A substituição do papel pelo produto contribuirá para a sustentabilidade ambiental. Além disso, evitam-se sobras de papel, i.e., as cópias da prova de avaliação que se tiraram a mais devido ao facto de o número de Alunos que apareceram para a realizar ser menor que o esperado.

Se assumirmos que:
\begin{itemize}
  \item um curso (de 3 anos) tem 6 unidades curriculares em cada semestre (ou seja 36 unidades curriculares);
  \item 90\% das provas de avaliação desse curso podem ser realizadas de forma digital;
  \item em cada unidade curricular se realizam 3 provas de avaliação, com uma média de 100 Alunos por prova;
\item cada Aluno consome 4 folhas de papel A4 em cada prova de avaliação em que participa;
\end{itemize}

então, em cada ano letivo, são poupadas 38.800 folhas de papel por curso, se for usado o produto Probum. Se cada impressão custar 0,05 EUR, então há uma poupança de quase 4.000 EUR (assume-se que cada folha é impressa em ambos os lados). Se uma universidade tiver o equivalente a 100 cursos destes, então a poupança cifra-se em 400.000 EUR anuais. Trata-se de um número bem expressivo que mostra a relevância deste produto. A isto há ainda que acrescentar os ganhos de tempo de correção, aspecto que não pode ser negligenciado e que consome muito do tempo de um professor nos períodos, por vezes longos, em que tem de corrigir provas.
%----------------------
\item \textbf{Consulta das provas:} Assim que todas classificações de uma dada prova forem divulgadas, o Docente poderá permitir que cada Aluno consulte a sua prova. Cada prova (que possa ser consultada) ficará disponível, no perfil do respetivo Aluno, para consulta, durante dois anos, altura em que poderá fazer sentido eliminá-la.

\end{itemize}
}

\hl{NOTA: tenho que validar estes dados com, pelo menos, algum diretor dum departamento universitário.}