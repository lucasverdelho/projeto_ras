\documentclass[pdftex,11pt,a4paper]{report}
\usepackage{graphicx}
\usepackage{subcaption}
\usepackage{float}
\usepackage{fancyvrb}
\fvset{xleftmargin=2em}
\usepackage{multicol}
\usepackage{wrapfig}
\usepackage{listings}
\usepackage[dvipsnames]{xcolor}
\usepackage{comment}
\usepackage{tikzscale}
\usepackage{pgfplotstable}
\usepackage{booktabs}
\usepackage[font=small,labelfont=bf,tableposition=top]{caption}
\usepackage{framed, tabularx} % para as requirement shells
\usepackage{graphicx}%
\newcommand{\authorblock}[1]{\begin{tabular}{@{}c@{}}#1\end{tabular}}

\usepackage[utf8]{inputenc}
\usepackage[portuges]{babel}
\usepackage[T1]{fontenc}
\usepackage{times}
\usepackage{lmodern}
\usepackage[obeyspaces,spaces]{url}
\usepackage[left=25mm,right=25mm,top=25mm,bottom=25mm]{geometry}
\usepackage{titlesec}
\usepackage{mathtools}
\usepackage{hyperref}

% FDias: packages e settings base para podermos usar o tikz para UML
\usepackage{amsmath}
\usepackage{tikz-uml}
\usetikzlibrary{positioning}
\tikzumlset{fill usecase=white}

% FDias: packages e settings base para podermos dar highlight a texto (para não me esquecer de rever certas partes)
\usepackage[normalem]{ulem}
\newcommand\hl{\bgroup\markoverwith
    {\textcolor{yellow}{\rule[-.5ex]{.1pt}{2.5ex}}}\ULon}

%identa 1º paragrafo de capitulos e secções
%\usepackage{indentfirst}
\setlength{\parindent}{0 mm}

\usepackage{eurosym}
\newcommand{\HRule}{\rule{\linewidth}{0.5mm}}
\titleformat{\chapter}{\normalfont\huge}{\thechapter.}{20pt}{\huge}

% FDias: define comandos para especificar priorização de requisitos (com cores)
\newcommand{\prioMust}{\noindent{\color{red}Must}}
\newcommand{\prioShould}{\noindent{\color{orange}Should}}
\newcommand{\prioCould}{\noindent{\color{blue}Could}}
\newcommand{\prioWont}{\noindent{\color{OliveGreen}Won't}}

\newcommand {\TextoCor}    {\color{teal}}

% FDias: define comando para criar tabela de requisitos
\newcommand{\req}[9]{
    \begin{table}[H]
    \begin{framed}
        \noindent
        \begin{minipage}{\linewidth}
            \begin{tabular}{p{0.25\textwidth}p{0.3\textwidth}p{0.45\textwidth}}
{\TextoCor Requisito \#: #1}   &   {\TextoCor
Tipo: #2}   &   {\TextoCor
\textit{Use cases} \#: #3}
            \end{tabular}
            
            \vspace{0.5cm}
            
            \begin{tabularx}{\textwidth}{rX}
                {\TextoCor Descrição}       & {\TextoCor \textbf{#4}} \\
                {\TextoCor \textit{Rationale}}  & {\TextoCor #5} \\
                {\TextoCor Origem}      & {\TextoCor #6} \\
                {\TextoCor \textit{Fit criterion}}  & {\TextoCor #7} \\
                {\TextoCor Prioridade}             & #8 \\
            \end{tabularx}
        \end{minipage}
    \end{framed}
    \caption{#9}
    \end{table}
}

\begin{document}
    \begin{titlepage}
	\centering
	\includegraphics[width=0.3\textwidth]{imagens/capa/uminho.png}\par\vspace{1cm}
	{\huge\bf UMinho\par}
	\vspace{0.5cm}
	{\huge\bf Mestrado Engenharia Informática\par}
	\vspace{0.1cm}
	{\huge\bf Requisitos e Arquiteturas de Software (2023/24)\par}
	\vspace{3cm}
    {\scshape\LARGE  \par}
    \vspace{0.5cm}
    {\scshape\Huge  PROBUM v.1\par}
    
    \vspace{2.5cm}
    
    \textbf{Paulo R. Sousa e João M. Fernandes}
    
    \vspace{5.5cm}
    
    \large Braga, {\large \today\par}

 \end{titlepage}


    \begin{abstract}
{\TextoCor
{\large Este documento surge na sequência do contacto do Reitor da Universidade de Vigo (Espanha) com o objetivo de criar um produto de software que permita a realização de provas de avaliação académicas.

Este produto, que se designa Probum, permite que alunos de uma dada unidade curricular de um curso universitário ou politécnico (i.e. do ensino superior) realizem as suas provas académicas, utilizando as infraestruturas informáticas da sua própria instituição de ensino superior (IES), mesmo que estas sejam muito limitadas quanto à sua dimensão, disponibilidade e capacidade. Assim, o Probum deve incluir requisitos funcionais que permitam a sua utilização em diversas IES  e permitir alguma parametrização e configuração.  No essencial, o Probum  permite que os professores criem provas de avaliação e que as calendarizem, que os alunos as realizem (de forma devidamente calendarizada) e que essas provas sejam corrigidas, tendencialmente de forma automática.} 
}
\vspace*{2cm}

\centering
{\TextoCor
\small Documento criado por Frederico Dias para a empresa XPTO.
}
\end{abstract}

    %\tableofcontents
    %\listoffigures
    %\listoftables
    \chapter*{1. Propósito do Projeto}

\section*{Contexto}



% Crescente digitalização do ensino, nomeadamente pilotos em provas da escolaridade obrigatória portuguesa
% Cursos com um número crecente de alunos inscritos, e UCs com várias provas de avaliação:
% - Complexidade na gestão de espaços para as provas (as salas são insuficientes para o número de alunos, pelo que uma ajuda informática seria uma ajuda muito importante);
% - Ineficiência na correção das provas (recursos e tempo);
% - Despredicio de papel.

{\TextoCor
Nos últimos anos, tem-se assistido a uma crescente integração da tecnologia no ambiente educativo e esse fenómeno não passa despercebido no ensino superior. Com o objetivo de otimizar processos e melhorar a experiência de Alunos e Docentes, muitas organizações têm explorado a digitalização de várias atividades, incluindo a realização de provas de avaliação.

Um exemplo interessante desta tendência é aquele que se observa em algumas provas de aferição do ensino básico português, que tentam explorar o potencial de transformação que as soluções digitais podem proporcionar.
No entanto, aplicar no imediato estes métodos numa instituição de ensino superior levanta novos desafios, que estão essencialmente relacionados com as limitações das estruturas informáticas. O crescente número de Alunos inscritos em muitas unidades curriculares é também um obstáculo à aplicação desses métodos.
Dentro desta realidade, destacam-se os seguintes desafios:

\begin{itemize}
\item \textbf{Gestão de espaços e recursos:} A realização de provas de avaliação para um grande número de Alunos coloca uma pressão significativa sobre os espaços disponíveis. As salas revelam-se muitas vezes insuficientes, resultando em desafios logísticos na calendarização e alocação destes espaços.

\item \textbf{Eficiência na correção:} A correção manual de um grande número de provas é um processo demorado e sujeito a erros humanos. Fazê-la, especialmente em cursos com muitos Alunos, torna-se ineficiente e onerosa em termos de tempo e custo.

\item \textbf{Sustentabilidade ambiental:} O uso de papel na impressão e resposta às provas de avaliação resulta num desperdício muito significativo, que tem um impacto negativo no ambiente. A necessidade de encontrar alternativas mais sustentáveis para a realização das provas é um aspeto crítico.

\end{itemize}
}

\section*{Objetivos do Projeto}
{\TextoCor
Para responder a estes desafios, propomos o desenvolvimento de um produto, o Probum, que tenha algumas características diferenciadoras e até inovadoras, que aborda todos estes problemas, relacionados com a realização de provas de avaliação, de forma integrada. O Probum permitirá:

\begin{itemize}
%----------------------
\item \textbf{Criação de provas de avaliação digitais:} Criar provas de avaliação personalizadas será simples e fácil e permitirá incorporar questões de diferentes tipos, temas e níveis de dificuldade. Deve ser possível criar questões alternativas, com respostas também diferenciadas, se for conveniente cada Aluno ter uma prova diferente das dos restantes colegas.
%----------------------
\item \textbf{Calendarização inteligente das provas:} Utilizando ferramentas matemáticas (e.g., algoritmos de otimização), o Probum poderá gerar automaticamente um calendário para uma dada prova de avaliação, tendo em consideração os espaços disponíveis, o número de Alunos inscritos, e as necessidades específicas de cada unidade curricular ou curso. Se uma dada prova tiver, por exemplo, 96 Alunos inscritos e a sala para a realização dessa prova só tiver capacidade para 20 Alunos, então têm que ser organizadas pelo menos 5 rondas de realização do teste. Este facto pode obrigar a que as provas não sejam iguais para todos os Alunos, como atrás se referiu.  
%----------------------
\item \textbf{Realização das provas:} Um Aluno poderá realizar as provas de avaliação de forma digital, através de uma plataforma dedicada, proporcionando assim uma experiência simples, robusta, e flexível. Esta plataforma será disponibilizada em espaços e equipamentos da própria IES, garantindo todos os requisitos exigidos (e.g., autenticidade, confidencialidade, equidade, duração, plágio). Deve ser analisada e explorada a possibilidade de compaginar a realização de provas em computadores da IES com provas realizadas em computadores dos Alunos, desde que se mantenham todas as garantias de confidencialidade e autenticidade.
%----------------------
\item \textbf{Correção:} Na fase de correção, todas as questões de resposta fechada serão facilmente avaliadas e pontuadas de forma automática. O Probum deverá poder ser estendido com componentes baseados em técnicas de processamento de linguagem natural, para auxiliar a correção de questões de resposta livre. Se tal funcionalidade não estiver disponível, cabe ao Docente pontuar essas questões abertas.
%----------------------
\item \textbf{Sustentabilidade:} A substituição do papel pelo produto contribuirá para a sustentabilidade ambiental. Além disso, evitam-se sobras de papel, i.e., as cópias da prova de avaliação que se tiraram a mais devido ao facto de o número de Alunos que apareceram para a realizar ser menor que o esperado.

Se assumirmos que:
\begin{itemize}
  \item um curso (de 3 anos) tem 6 unidades curriculares em cada semestre (ou seja 36 unidades curriculares);
  \item 90\% das provas de avaliação desse curso podem ser realizadas de forma digital;
  \item em cada unidade curricular se realizam 3 provas de avaliação, com uma média de 100 Alunos por prova;
\item cada Aluno consome 4 folhas de papel A4 em cada prova de avaliação em que participa;
\end{itemize}

então, em cada ano letivo, são poupadas 38.800 folhas de papel por curso, se for usado o produto Probum. Se cada impressão custar 0,05 EUR, então há uma poupança de quase 4.000 EUR (assume-se que cada folha é impressa em ambos os lados). Se uma universidade tiver o equivalente a 100 cursos destes, então a poupança cifra-se em 400.000 EUR anuais. Trata-se de um número bem expressivo que mostra a relevância deste produto. A isto há ainda que acrescentar os ganhos de tempo de correção, aspecto que não pode ser negligenciado e que consome muito do tempo de um professor nos períodos, por vezes longos, em que tem de corrigir provas.
%----------------------
\item \textbf{Consulta das provas:} Assim que todas classificações de uma dada prova forem divulgadas, o Docente poderá permitir que cada Aluno consulte a sua prova. Cada prova (que possa ser consultada) ficará disponível, no perfil do respetivo Aluno, para consulta, durante dois anos, altura em que poderá fazer sentido eliminá-la.

\end{itemize}
}

\hl{NOTA: tenho que validar estes dados com, pelo menos, algum diretor dum departamento universitário.}
    %----------------------------------------------
\chapter*{2. Cliente, Consumidor e \textit{Stakeholders}}

{\TextoCor

Ao longo do desenvolvimento deste projeto verificamos a existência de diversas partes envolvidas, nomeadamente: clientes, consumidores e \textit{stakeholders}.

\begin{itemize}
\item \textbf{Clientes:} Os nossos clientes serão os departamentos pedagógicos de IESs com a necessidade de disponibilizar aos seus corpos docente e discente um sistema informático que automatize a criação, calendarização, realização e correção de provas de avaliação (escritas e individuais).

\item \textbf{Consumidores:} Os consumidores do nosso produto são os docentes de ensino superior. Eles estão interessados numa ferramenta informática que facilite a criação, distribuição e correção das provas de avaliação, ao mesmo tempo que os mantém capacitados para a resolução de situações anómalas durante as suas realizações. Será através dos docentes que devemos definir como deverá ser a dinâmica de criação, calendarização, realização e correção das provas de avaliação. Durante o desenvolvimento do projeto, os nossos clientes participarão de forma ativa no planeamento e validação da solução.

\item \textbf{Outros \textit{stakeholders}} 
    \begin{itemize}
        \item Alunos: Uma das principais partes interessadas no nosso produto são os estudantes universitários, que engloba todas as pessoas que estão inscritas num curso superior e que têm que realizar provas de avaliação. Pode considerar-se que as suas idades são superiores (ou iguais) a 18 anos, e que estão familiarizados com o uso de ferramentas informáticas. Deve ser-lhes disponibilizado um produto simples, eficiente, robusto, amigável, e tolerante a falhas para que possam realizar as suas provas.
        
        \item Técnicos: Os técnicos de informática são também fundamentais para o funcionamento do produto. São eles que instalam e fazem a manutenção da plataforma nos equipamentos informáticos da IES. Eles esperam que o produto seja fácil de instalar e configurar, e que disponibilize métricas e \textit{logs} que permitam antecipar e diagnosticar eventuais problemas.
    \end{itemize}
\end{itemize}

}
    %----------------------------------------------
\chapter*{3. Utilizadores do Produto}

{\TextoCor
Nesta secção apresentam-se todos os utilizadores que utilizarão efetivamente o produto, listando-se as funções de cada um.
}
\subsection*{Docente}
{\TextoCor
\begin{itemize}
    \item \textbf{Função:} Responsável por contactar com os alunos, criar provas de avaliação, garantir que as mesmas se realizam nos espaços da sua IES e coordenar o processo de correção (quando necessário).
    \item \textbf{Experiência no contexto:} mestre
    \item \textbf{Experiência tecnológica:} mediana
\end{itemize}
}

\subsection*{Aluno}
{\TextoCor
\begin{itemize}
    \item \textbf{Função:} Responsável por dar respostas às questões das provas de avaliação em que se apresentar.
    \item \textbf{Experiência no contexto:} mediana
    \item \textbf{Experiência tecnológica:} mediana
\end{itemize}
}

\subsection*{Técnico}
{\TextoCor
\begin{itemize}
    \item \textbf{Função:} responsável por instalar a plataforma nos equipamentos informáticos da IES e mantê-la em funcionamento.
    \item \textbf{Experiência no contexto:} mediana
    \item \textbf{Experiência tecnológica:} mestre
\end{itemize}
}

\section*{Prioridades atribuídas aos Utilizadores}
{\TextoCor
\begin{itemize}
    \item \textbf{Utilizadores principais:} Docente e Aluno
    \item \textbf{Utilizadores secundários:} Técnico
\end{itemize}

O sucesso do produto depende diretamente dos Docentes e dos Alunos, e, por esse motivo, é nas suas necessidades e expectativas que se deve focar o esforço de levantamento de requisitos.
Os Técnicos têm relevância nos requisitos, no entanto, caso estes colidam com os dos Docentes e Alunos, será dada preferência a estes.
}
    \chapter*{4. Restrições do Projeto}  

\section*{Restrições à Solução}

{\TextoCor

\req
    {Rest1} % ID requisito
    {Restrição} % tipo de requisito
    {n.a.} % use cases
    {A aplicação deve executar na infraestrutura atual da respetiva IES} % descricao
    {Para que não seja necessário investir em novo equipamento} % rational
    {Cliente} % origem
    {Todos os componentes de software devem estar instalados em máquinas da IES e todas as funcionalidades da plataforma para os Alunos devem executar em pleno nas máquinas que que forem disponibilizadas para as provas de avaliação} % fit criterion
    {\prioMust} % prioridade
    {Restrição quanto à infraestrutura informática} % caption

\req
    {Rest2} % ID requisito
    {Restrição} % tipo de requisito
    {n.a.} % use cases
    {O computador disponibilizado a cada Aluno, no momento em que realiza uma prova de avaliação, apenas deve permitir acesso ao Probum} % descricao
    {Para evitar que o Aluno recorra a outras aplicações (email, navegadores web, WhatsApp, Skype, etc.) durante a realização da sua prova de avaliação; O produto tem que estar preparado para funcionar em computadores instalados em salas da IES } % rational
    {Cliente} % origem
    {Enquanto uma prova de avaliação estiver a decorrer não deve ser possível aceder a nenhuma outra aplicação que não o Probum} % fit criterion
    {\prioMust} % prioridade
    {Restrição quanto ao isolamento da aplicação de resposta às provas} % caption

}

\section*{Restrições Temporais}
{\TextoCor
\begin{itemize}
    \item \textbf{Descrição:} O documento presente terá de ser entregue, numa fase inicial, até dia 20 de outubro de 2023.
    
    \textbf{Justificação:} De forma a poder ser avaliado o estado do projeto numa fase inicial, é necessário que seja feita uma entrega que contenha a primeira fase deste projeto, que abarca a contextualização e a definição dos requisitos da solução.
\end{itemize}
}

\section*{Restrições Orçamentais}
{\TextoCor
\begin{itemize}
    \item \textbf{Descrição:} O orçamento total para o desenvolvimento do projeto é de 20 000€ (vinte mil euros), durante um período de 4 meses.
    
    \textbf{Justificação:} A equipa responsável pelo desenvolvimento do projeto é constituída por quinze engenheiros de \textit{software}. Para além de ter em conta os salários dos elementos, é preciso também a compra de um domínio, bem como  de um computador para hospedar todos os dados da aplicação.
\end{itemize}
}
    \chapter*{5. Taxonomia e definições}
\newcommand{\define}[2]{\item[#1] \hfill \\ #2}


\begin{description}    
{\TextoCor
    \define{Aluno}
    {Ator do sistema. Responsável por se autenticar e responder às Questões de uma Prova; O mesmo que estudante ou discente.}
    
    \define{Classificação}
}
    {...}

{\TextoCor
    \define{Computador}
    {Equipamento informático no qual um Aluno dá as Respostas às Questões da sua Prova. Ainda que eventualmente intermitente, estes equipamentos têm ligação à \textit{intranet} da IES.}

    \define{Correção}
}
    {...}
    
{\TextoCor
    \define{Critério de Avaliação}
}    
    {...}
    
 {\TextoCor
   \define{Docente}
    {Ator do sistema. Responsável por criar e dar seguimento a uma prova. Na criação de uma Prova, entre outras tarefas, define as Questões e os respetivos Critérios de Avaliação. O mesmo que professor.}
    
    \define{Instituição de Ensino Superior}
    {Organização de ensino que tem interesse em disponibilizar a plataforma aos seus docentes, para que seja possível utilizar a infraestrutura informática já existente para a realização de provas de avaliação; O mesmo que Universidade ou Instituto Politécnico; Sigla: IES.}

    \define{Prova de avaliação}
    {Avaliação de competências e de conhecimentos, em que cada Aluno responde, de forma individual e durante um período de tempo previamente estabelecido, a um conjunto de Questões, que foram previamente preparadas pelos Docentes; O mesmo que teste escrito, prova escrita ou exame.}
}

{\TextoCor
    \define{Questão}
}    
    {...}
    
{\TextoCor
    \define{Resposta}
}    
    {...}

 {\TextoCor
   \define{Sala}
   
    {Espaço físico equipado com computadores, onde se realizam as Provas. Geralmente têm capacidade para um número reduzido de Alunos em simultâneo  (por exemplo, cerca de 20).}
}

\end{description}


    \chapter*{8. Âmbito do Produto}
\section*{Diagrama de \textit{Use Cases}}
{\TextoCor
De maneira a compreender melhor o contexto do sistema, vai ser apresentado um diagrama de \textit{Use Cases}. Neste vão ser explicitadas algumas das principais funcionalidades do sistema, bem como os atores do mesmo.
Neste diagrama é ainda possível identificar a que funcionalidades cada ator do sistema terá acesso.
}
\begin{figure}[H]
    \centering
    \begin{tikzpicture}
        \begin{umlsystem}[x=6, fill=cyan!20] {Probum}
            \umlusecase[name=testar_plataforma] {Testar plataforma}
            
            % FDias: VER! aqui ainda não tenho a certeza se o registo de alunos deve ser um use case ou não.
            % Se o registo de alunos for feito à UC, e até puder ser feito pelos Técnicos, ou até pelos Serviços Académicos de forma automática, faz sentido.
            % Mas se este registo for algo mais ad-hoc, em cada prova, acho que devia passar para o use case de "Criar Prova".
            % Temos de esclarecer este ponto!
            \umlusecase[name=registar_alunos, below=0.5cm of testar_plataforma] {\hl{Registar Alunos}}
            \umlusecase[name=criar_prova, below=0.5cm of registar_alunos] {Criar Prova}
            \umlusecase[name=iniciar, below=0.5cm of criar_prova] {Iniciar Prova}
            \umlusecase[name=finalizar, below=0.5cm of iniciar] {Finalizar Prova}
            \umlusecase[name=classificar, below=0.5cm of finalizar] {Classificar Respostas}
            \umlusecase[name=responder, below=0.5cm of classificar] {Responder a Prova}
            \umlusecase[name=consultar_prova, below=0.5cm of responder] {Consultar Prova}
        \end{umlsystem}
    
        \umlactor[y=0, y=-4] {Docente}
        \umlactor[x=12] {Técnico}
        \umlactor[x=12, y=-3] {Sistema}
        \umlactor[x=12, y=-7] {Aluno}
    
        \umlassoc{Docente}{criar_prova}
        \umlassoc{Docente}{registar_alunos}
        \umlassoc{Docente}{iniciar}
        \umlassoc{Docente}{finalizar}
        \umlassoc{Docente}{classificar}
        \umlassoc{Docente}{testar_plataforma}
        \umlassoc{Docente}{registar_alunos}
        
        \umlassoc{Aluno}{responder}
        \umlassoc{Aluno}{consultar_prova}
        
        \umlassoc{Técnico}{testar_plataforma}
        \umlassoc{Técnico}{registar_alunos}
        
        \umlassoc{Sistema}{classificar}
        \umlassoc{Sistema}{testar_plataforma}
        \umlassoc{Sistema}{registar_alunos}
    \end{tikzpicture}
    \caption{Diagrama de \textit{Use Cases}}
\end{figure}

% FDias: temos de encaixar toda a parte de calendarização automática

\newpage


\section*{Atores}
{\TextoCor
Como está representado no diagrama anterior, o nosso sistema suporta quatro tipos de atores: o \textit{Docente}, o \textit{Técnico}, o \textit{Aluno}, e o \textit{Sistema}. 

O \textit{Docente} é o utilizador central do nosso sistema, pois é ele que vai gerir todo o conteúdo e funcionamento da plataforma.

O \textit{Aluno}, por sua vez, é o utilizador mais crítico e sensível do sistema, dado o contexto em que utilizará a plataforma.

O \textit{Técnico} é o indivíduo que ajudará a instalar, configurar, e manter a plataforma nos servidores e Computadores da IES.

O \textit{Sistema} é uma ou mais peças de \textit{software} cujo propósito é automatizar processos e tarefas, tais como a correção total ou parcial de Respostas.
}



\section*{Breve Descrição dos \textit{Use Cases}}
{\TextoCor
Esta secção apresenta uma especificação tabelar de cada \textit{use case} considerado, de modo a facilitar o processo de implementação de cada funcionalidade do nosso sistema.

Deste modo, consideramos que é bastante percetível o fluxo sequencial da interação do ator com o sistema.

\newcounter{useCases}
\setcounter{useCases}{0}

\newcommand{\descCenario}[3]{
    \begin{tabular}[c]{@{}l@{}}\textbf{#1}\\ {[}#2]\\ (#3){]}\end{tabular}
}
}

\subsection*{Criar Prova}

{\TextoCor
O \textit{use case} "Criar Prova", cujo ator principal é o \textit{Docente}, consiste na criação de uma nova prova de avaliação.
Esta criação implica a introdução de detalhes básicos acerca da prova (e.g., local, date e hora, duração), mas também contempla o registo de alunos e a criação de questões.

Existe um cenário de exceção quando o sistema não consegue interpretar o ficheiro submetido pelo Docente. Nesse caso são devolvidos os detalhes do erro, para que o ficheiro possa ser retificado e posteriormente ressubmetido.
}

\begin{table}[H]
    \begin{center}
        \scalebox{0.85}{
            \begin{tabular}{|l|l|}
                \hline
                \textbf{\textit{Use case}} & \stepcounter{useCases}\theuseCases \\ \hline
                \textbf{Ator principal} & Docente \\ \hline
                \textbf{Ator secundário} & - \\ \hline
                \textbf{Pré-Condições} & O Docente está autenticado na plataforma \\ \hline
                \textbf{Pós-Condições} & A Prova está criada no sistema, e conta com \hl{Alunos} e Questões registadas \\ \hline
                \textbf{Cenário Normal} & \begin{tabularx}{13cm}{X|X}
                    \textbf{Input do Ator} & \textbf{Resposta do Sistema} \\ \hline
                    1 - O Docente fornece data, \hl{hora}, sala e outros detalhes & \\

                    % FDias: Ver questão da inscrição dos alunos e uniformizar! (comentário na figura dos Diagramas de Use Cases)
                    % FDias: também a parte de calendarização automática
                    & 2 - O sistema cria a Prova com todos os detalhes, mas ainda incompleta (e.g., sem Questões \hl{e Alunos}) \\
                    3 - O Docente submete um ficheiro com a informação dos Alunos a registar na Prova & \\
                    & 4 - O Sistema valida o ficheiro submetido \\
                    & 5 - O Sistema associa os Alunos à Prova \\
                    6 - O Docente cria uma Questão, selecionando o seu tipo e a sua descrição & \\
                    7 - O Docente define restrições às Respostas (e.g., escolha múltipla, V/F, resposta aberta) & \\
                    8 - O Docente define os Critérios de Avaliação da Questão & \\
                    & 9 - O Sistema cria a Questão \\
                    10 - O Docente volta ao passo 6 &  \\
                \end{tabularx} \\ \hline
                
                \descCenario{Exceção 1}{formato inválido}{Passo 4} & \begin{tabularx}{13cm}{X|X}
                    & 4.1 - O Sistema informa o Docente que o ficheiro não é valido, e dá informação suficiente para que o problema seja corrigido\\
                \end{tabularx} \\ \hline

                \descCenario{Alternativa 1}{terminado}{Passo 10} & \begin{tabularx}{13cm}{X|X}
                    10 - O Docente dá por terminada a criação da Prova & \\
                \end{tabularx} \\ \hline
            \end{tabular}
        }
        \caption{Especificação do \textit{use case} "Criar Prova"}
    \end{center}
\end{table}

\subsection*{...}

    \chapter*{9. Requisitos Funcionais}


\section*{Modelação de Requisitos}
{\TextoCor
Para o levantamento de requisitos foi utilizada a \textit{requirement shell} do modelo \textit{Volere} como forma de representação, para os descrever concisamente.
}

\req
    {12} % ID requisito
    {Funcional} % tipo de requisito
    {1, 7, 20} % use cases
    {O produto deve registar todas as estradas que foram repavimentadas} % descricao
    {Para permitir agendar o reparo de estradas não repavimentadas e identificar potenciais perigos} % rational
    {Equipa} % origem
    {O registo das estradas repavimentadas deve seguir a especificação da IP e ser feito até 30 minutos após concluída a repavimentação da estrada} % fit criterion
    {\prioMust} % prioridade
    {Exemplo de especificação de um requisito} % caption


{\TextoCor
Como caracterização da tabela de representação de requisitos, é necessário descrever os campos:
\begin{itemize}
    \item \textbf{Requisito:} número de identificação do requisito.
    \item \textbf{Tipo:} tipo de requisito, considerando o modelo de \textit{Volere}.
    \item \textbf{\textit{Use Cases}:} número dos Use Cases associados.
    \item \textbf{Descrição:} descrição clara e concisa do requisito.
    \item \textbf{\textit{Rationale}:} razão para a existência do requisito.
    \item \textbf{Origem:} quem originou o requisito.
    \item \textbf{\textit{Fit criterion}:} critério para validar cumprimento do requisito.
    \item \textbf{Prioridade:} índice de prioridade para a implementação do requisito:
    \begin{itemize}
        \item \prioMust: requisito obrigatório;
        \item \prioShould: requisitos que deve ser implementados;
        \item \prioCould: requisito que não é necessário, mas é desejado;
        \item \prioWont: requisito que pode ser considerado posteriormente.
    \end{itemize}
    \item \textbf{Data:} data da especificação do requisito.
\end{itemize}
}

\section*{Requisitos Funcionais}

{\TextoCor

\req
    {Req1} % ID requisito
    {Funcional} % tipo de requisito
    {1} % use cases
    {O Docente cria uma prova de avaliação} % descricao
    {Para permitir associar questões que serão respondidas pelos alunos e, posteriormente, classificadas} % rational
    {Cliente} % origem
    {A criação de uma prova implica introduzir toda a informação necessária à sua realização e inequívoca identificação} % fit criterion
    {\prioMust} % prioridade
    {Requisito funcional quanto à criação de uma prova de avaliação} % caption

% FDias: Ver questão da inscrição dos alunos e uniformizar! (comentário na figura dos Diagramas de Use Cases)
\req
    {Req2} % ID requisito
    {Funcional} % tipo de requisito
    {1} % use cases
    {\hl{O Técnico regista Alunos numa prova de avaliação}} % descricao
    {Para permitir definir que Alunos poderão participar na prova, associando-lhes também um método de autenticação. Em alternativa, o Docente também deve poder registar alunos, que não estejam, por alguma razão, ainda inscritos.} % rational
    {Cliente} % origem
    {O processo de registo de alunos deve permitir definir que alunos podem participar na prova, e garantir que cada um deles pode ser autenticado pela plataforma aquando da sua realização} % fit criterion
    {\prioMust} % prioridade
    {Requisito funcional quanto ao registo de alunos numa prova de avaliação} % caption

\req
    {Req3} % ID requisito
    {Funcional} % tipo de requisito
    {1} % use cases
    {O Docente adiciona Questões de escolha múltipla a uma prova de avaliação} % descricao
    {Para permitir avaliar o conhecimento dos Alunos numa dada temática, facultando um conjunto de possíveis respostas, onde apenas uma é a correta} % rational
    {Cliente} % origem
    {As Provas de avaliação podem ser compostas por Questões de escolha múltipla} % fit criterion
    {\prioMust} % prioridade
    {Requisito funcional quanto à possibilidade de criar Questões de escolha múltipla} % caption

\req
    {Req4} % ID requisito
    {Funcional} % tipo de requisito
    {\hl{...}} % use cases
    {O Docente define o momento a partir do qual as provas podem ser consultadas} % descricao
    {Para permitir ao Docente disponibilizar as respostas dadas por cada Aluno, assim como as respetivas classificações} % rational
    {Cliente} % origem
    {As Provas de avaliação apenas podem ser consultadas após o momento definido pelo Docente} % fit criterion
    {\prioShould} % prioridade
    {Requisito funcional quanto à possibilidade de disponibilizar Provas de avaliação para consulta} % caption
}

% \req
%     {5} % ID requisito
%     {Funcional} % tipo de requisito
%     {\hl{...}} % use cases
%     {O produto deve permitir definir o momento a partir do qual as provas podem ser consultadas} % descricao
%     {Para permitir ao docente disponibilizar as respostas dadas por cada aluno, assim como as respetivas classificações} % rational
%     {Cliente} % origem
%     {As provas de avaliação apenas podem ser consultadas após o momento definido pelo docente} % fit criterion
%     {\prioShould} % prioridade
%     {Requisito funcional quanto à possibilidade de disponibilizar provas para consulta} % caption

% FDias: algumas ideias
% - Alunos poderem ter acesso (ou até receber) um PDF com as suas respostas / classificações
% - Prof. poder usar bold, itálico, tamanhos de texto diferentes, etc. nas provas
% - Prof. poder associar regras em regex para validar input dos alunos (se a resposta tiver de ser um número, validar que é efetivamente um número logo no UI qd o aluno está a escrever)
% - Prof. poder editar a prova enquanto ela está a decorrer (algum erro que seja detetado). Não podemos é afetar as respostas já dadas: imutabilidade / versões (?)
% - O UI dos Alunos poder mostrar um cronómetro (configurável pelo aluno [?])
% - Prof. poder dar indicação, junto da questão, o tempo esperado para a resposta àquela questão. Depois apresentar isso ao aluno (tb configurável) (?)
% - Prof. poder adicionar multimedia às questões
% - 
% - ...
% Nota: temos de fazer algumas entrevistas com Profs. e Alunos para percebermos as necessidades.

...

    \chapter*{10--17. Requisitos Não Funcionais}

% tem de ficar no capítulo 16
\req
    {RNF1} % ID requisito
    {\textit{Cultural and political}} % tipo de requisito
    {\hl{...}} % use cases
    {O produto está preparado para suportar vários idiomas (e.g., português, espanhol, inglês)} % descricao
    {Para permitir que os seus utilizadores o possam usar utilizando a língua que mais dominam} % rational
    {Cliente} % origem
    {Utilizadores que dominam diferentes idiomas têm a mesma experiência de utilização} % fit criterion
    {\prioShould} % prioridade
    {Requisito não funcional cultural e político quanto ao produto suportar vários idiomas} % caption


\hl{NOTA:} 

\begin{itemize}
\item 
\hl{Probum tem de correr em computadores de gama baixa.}

\item
\hl{Probum deve dar suporte a várias línguas}.

\item
\hl{As Respostas dadas pelos Alunos nas Provas de avaliação têm que ficar disponíveis durante 2 anos, para permitir tratar eventuais reclamações.}

























\newpage
\section{Aparência}

\newpage
\section{Usabilidade}

\newpage
\section{Performance}

\newpage
\section{Operacional}


\newpage
\section{Manutenção e Suporte}


\newpage
\section{Segurança}


\newpage
\section{Cultural e Político}

\newpage
\section{Legal}




\end{itemize}
    \chapter*{21. Levantamento de Requisitos}

\hl{Fazer um pequeno resumo relativo às técnicas que foram usadas para recolher/levantar os requisitos. Assumi que isto faz parte das "Tasks", más é discutível.}


- entrevistas 
- surveys
- introspeção 
- use cases
- cenários
- 
\end{document}