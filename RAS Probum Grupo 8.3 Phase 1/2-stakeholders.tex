%----------------------------------------------
\chapter*{2. Cliente, Consumidor e \textit{Stakeholders}}

{\TextoCor

Ao longo do desenvolvimento deste projeto verificamos a existência de diversas partes envolvidas, nomeadamente: clientes, consumidores e \textit{stakeholders}.

\begin{itemize}
\item \textbf{Clientes:} Os nossos clientes serão os departamentos pedagógicos de IESs com a necessidade de disponibilizar aos seus corpos docente e discente um sistema informático que automatize a criação, calendarização, realização e correção de provas de avaliação (escritas e individuais).

\item \textbf{Consumidores:} Os consumidores do nosso produto são os docentes de ensino superior. Eles estão interessados numa ferramenta informática que facilite a criação, distribuição e correção das provas de avaliação, ao mesmo tempo que os mantém capacitados para a resolução de situações anómalas durante as suas realizações. Será através dos docentes que devemos definir como deverá ser a dinâmica de criação, calendarização, realização e correção das provas de avaliação. Durante o desenvolvimento do projeto, os nossos clientes participarão de forma ativa no planeamento e validação da solução.

\item \textbf{Outros \textit{stakeholders}} 
    \begin{itemize}
        \item Alunos: Uma das principais partes interessadas no nosso produto são os estudantes universitários, que engloba todas as pessoas que estão inscritas num curso superior e que têm que realizar provas de avaliação. Pode considerar-se que as suas idades são superiores (ou iguais) a 18 anos, e que estão familiarizados com o uso de ferramentas informáticas. Deve ser-lhes disponibilizado um produto simples, eficiente, robusto, amigável, e tolerante a falhas para que possam realizar as suas provas.
        
        \item Técnicos: Os técnicos de informática são também fundamentais para o funcionamento do produto. São eles que instalam e fazem a manutenção da plataforma nos equipamentos informáticos da IES. Eles esperam que o produto seja fácil de instalar e configurar, e que disponibilize métricas e \textit{logs} que permitam antecipar e diagnosticar eventuais problemas.
    \end{itemize}
\end{itemize}

}